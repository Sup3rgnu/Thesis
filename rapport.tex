\documentclass[a4paper,11pt]{kth-mag}
\usepackage[T1]{fontenc}
\usepackage{textcomp}
\usepackage{lmodern}
\usepackage[utf8]{inputenc}
\usepackage[swedish, english]{babel}
\usepackage{modifications}
\usepackage{hyperref}
\usepackage{listings}
\usepackage{color}
\definecolor{gray}{rgb}{0.4,0.4,0.4}
\definecolor{darkblue}{rgb}{0.0,0.0,0.6}
\definecolor{cyan}{rgb}{0.0,0.6,0.6}


\lstset{
  basicstyle=\ttfamily,
  columns=fullflexible,
  showstringspaces=false,
  commentstyle=\color{gray}\upshape,
  captionpos=b
}

\lstdefinelanguage{XML}
{
  morestring=[b]",
  morestring=[s]{>}{<},
  morecomment=[s]{<?}{?>},
  stringstyle=\color{black},
  identifierstyle=\color{darkblue},
  keywordstyle=\color{cyan},
  morekeywords={xmlns,version,type}% list your attributes here
}

\title{Semantic parsing of Swedish laws}

\subtitle{Using RDF Lorem ipsum dolor sit amet, sed diam nonummy nibh eui
              mod tincidunt ut laoreet dol}
\foreigntitle{Lorem ipsum dolor sit amet, sed diam nonummy nibh eui
              mod tincidunt ut laoreet dol}
\author{Mikael Falgard}
\date{\today}
\blurb{Master's Thesis at NADA\\Supervisor: Sten Andersson\\Examiner: Anders Lansner}
\trita{TRITA xxx yyyy-nn}
\begin{document}
\frontmatter
\pagestyle{empty}
\removepagenumbers
\maketitle
\selectlanguage{english}
\begin{abstract}
  This is a skeleton for KTH theses. More documentation
  regarding the KTH thesis class file can be found in
  the package documentation.

Lorem ipsum dolor sit amet, consectetuer adipiscing elit. Mauris
purus. Fusce tempor. Nulla facilisi. Sed at turpis. Phasellus eu
ipsum. Nam porttitor laoreet nulla. Phasellus massa massa, auctor
rutrum, vehicula ut, porttitor a, massa. Pellentesque fringilla. Duis
nibh risus, venenatis ac, tempor sed, vestibulum at, tellus. Class
aptent taciti sociosqu ad litora torquent per conubia nostra, per
inceptos hymenaeos.
\end{abstract}
\clearpage
\begin{foreignabstract}{swedish}
  Denna fil ger ett avhandlingsskelett.
  Mer information om \LaTeX-mallen finns i
  dokumentationen till paketet. ÅÄÖ lalal

Lorem ipsum dolor sit amet, consectetuer adipiscing elit. Mauris
purus. Fusce tempor. Nulla facilisi. Sed at turpis. Phasellus eu
ipsum. Nam porttitor laoreet nulla. Phasellus massa massa, auctor
rutrum, vehicula ut, porttitor a, massa. Pellentesque fringilla. Duis
nibh risus, venenatis ac, tempor sed, vestibulum at, tellus. Class
aptent taciti sociosqu ad litora torquent per conubia nostra, per
inceptos hymenaeos.
\end{foreignabstract}
\clearpage
\tableofcontents*
\mainmatter
\pagestyle{newchap}

\part{Introduction}
\chapter{Introduction}

\section{Swedish Laws}

According to \textit{Rättsinformationsförordningen} (1999:175) basic legal information has to be provided both the public administration, and to private individuals in electronic form. This is done by the Government Offices legal databases. These databases are old and hard to navigate since they only consist of plain text documents.

\section{Rättsinformationssystemet}
Lorem Ipsum

\section{Rättsnätet} Rättsnätet is a free web service on
\url{www.notisum.se}\footnote{Notisum AB is the company that I am writing this thesis at.} that provides legal
information. Rättsnätet’s content consists of information gathered from
authority databases and is processed in several steps from pure text information
into a rich XML file. Notisum also provides a premium service that includes
additional information and services.

\section{Lagen.nu}
A similar website to Notisum is lagen.nu, they also provide all Swedish laws as a free web service. A difference is that lagen.nu follows standards regarding the Semantic Web\footnote{The Semantic Web is a collaborative movement that promotes common data formats on the World Wide Web.} in a better way than Notisum's Rättsnätet does. Lagen.nu is created by an individual with voluntary interest in law and the Semantic Web. The source code is published as open source, and the website encourages sharing and reuse of the code. 

\section{Problem}
Today Rättsnätet use processing steps that downloads, analyzes and transforms authority information from plain text to structured and marked up XML information. This process is a collection of programs written in Delphi Pascal and C\#. These programs, especially those that have been written in Delphi Pascal, are old and difficult to maintain. The conversion parts are constructed using a traditional technique similar to that used to write compilers.\\\\
Instead of using compiler technology lagen.nu use regular expressions to parse law documents. The conversion programs lagen.nu use are written in object-oriented Python, which is more suitable for the task than Delphi Pascal that Notisum use today. Additionally lagen.nu is based on a document model from the Semantic Web, with the Resource Description Framework, RDF, as a base. Because of this the code base in lagen.nu has greater flexibility and is more about standards than Notisum.\\\\
The problem that this thesis focus on is how to modernize and replace Rättsnätet's current platform to a more modern process. A process where the result should be based on the Semantic Web and follow standards proposed by Rättsinformationsprojektet.

\subsection{Questions of concern}

TODO:

\begin{itemize} 
\item The amount of documents. There is over 200,000 documents, that amount requires a very fast process. Today the PP does its job in 14 hours a day, but then it only handles a tenth of the documents per day. 
\item Legal source text contains references that should be interpreted differently depending on the context, and to be marked up automatically with a certain margin of error.
\item Cross-references between the 200,000+ documents require theoretically, an array of 40 billion nodes.
\item The output from the process needs to be structurally similar to the current process output so the post process can be used with as little tweaking as possible.
\end{itemize}

\section{Limitations}
Lorem ipsum dolor sit amet, consectetuer adipiscing elit. Mauris
purus. 

\section{Abbreviations and Vocabulary} To be able to understand this report,
the reader needs to understand the following abbreviations. Given the
complexity of a legal document\footnote{For example "a paragraph" in a Swedish
legal document does not correspond to a regular paragraph in the English
language.} I will also present a vocabulary of how words are defined in the
thesis.

\subsection*{SFS} The Swedish Code of Statutes “Svensk författningssamling”
(SFS) is the official publication of all Swedish laws enacted by the Riksdag
and ordinances issued by the Government\footnote{SFS -
\url{http://en.wikipedia.org/wiki/Swedish_Code_of_Statutes}}. Every law and
ordinance has an SFS number, it consists of a four digit year, a colon, and
then an incrementing number by year. For instance the Ordinance on tattooing
dyes have the SFS number 2012:503\footnote{PDF copy of 2012:503 -
\url{http://notisum.se/rnp/sls/sfs/20120503.pdf}}.

\subsection*{SFSR}
The Statute Register (SFSR) includes registry information of Swedish Code
of Statutes (SFS), amendments, references to preparatory work, etc.

\subsection*{SFST}
Statutes in full text (SFST) includes Swedish Code of Statutes (SFS) in
full text, ie all applicable laws and regulations.

\subsection*{URI} Short for Unified Resource Identifier. Uniquely identifies
resources in a collection\footnote{URI - \url{http://www.w3.org/TR/uri-
clarification/}}, for example an hypertext transfer protocol url specifies a
webpage on the internet.

\subsection*{XML} Extensible Markup Language (XML) defines a set of rules for
encoding documents in a format that is readable for both humans and
machines\footnote{XML - \url{http://en.wikipedia.org/wiki/XML}}.

\subsection*{HTML} Hypertext Markup Language (HTML) is the main language for
displaying web pages and other information that can be displayed in a web
browser\footnote{HTML - \url{http://en.wikipedia.org/wiki/Html}}.

\subsection*{RDF} Resource Description Framwork (RDF) is a general method to
describe or model information that is implemented in web
resources\footnote{More detailed information about RDF in the ‘Background’
section}.

\subsection*{UTF-8} A encoding that can represent every character in the
Unicode character set. UTF-8 has become the dominant character encoding for
the World-Wide Web.

\subsection*{ISO-8859-1} A character encoding that is generally intended for
"Western European" languages, it encodes what it refers to as "Latin alphabet
no 1" consisting of 191 characters from the Latin script.

\subsection*{N3}
Notation3, or N3 is a shorthand non-XML serialization of RDF models,
designed with human-readability in mind which makes it more compact and
readable than XML RDF notation.

\subsection*{Dublin Core}
The Dublin Core metadata terms are a set of vocabulary terms which
can be used for simple resource description, or combining metadata
vocabularies of different metadata standards in Semantic web implementations.

\chapter{Background}

Lorem ipsum dolor sit amet, consectetuer adipiscing elit. Mauris
purus. 

\section{Swedish law}
One of the most fundamental definitions in this paper is legal sources and legal source documents. A legal source is a type of legal information such as a constitution in SFS\footnote{The Swedish Code of Statutes, "Svensk författningssamling"} or verdicts from one of the Swedish courts. A legal source document is a specific document for a certain legal source, such as \textit{Personuppgiftslagen} (SFS 1998:204).\\\\ 
This paper will be limited to content from SFS, a abbreviation for "Svensk författningssamling". SFS is the official publication of all new Swedish laws enacted by the Swedish Parliament and ordinances issued by the Government. Each law has an SFS number, including legislations amending already existing laws. The SFS number consists of a four digit year, a colon and a serial number assigned in chronological order of the date of issue. 

\section{Parsing laws}
Lorem ipsum dolor sit amet, consectetuer adipiscing elit. Mauris
purus. 

\section{Existing system}
Lorem ipsum dolor sit amet, consectetuer adipiscing elit. Mauris
purus. 

\section{Generic model}
Lorem ipsum dolor sit amet, consectetuer adipiscing elit. Mauris
purus. 

\section{The Semantic Web} Natural languages have ambiguous meanings and even
a human reader may in some cases have problems understanding the correct
meaning of a text (for example the question "Do you know Bush?" may refer to
knowing him as a friend or knowing who he is). By adding labels to text we
create a semantic web, formed so that software can collect data.

The semantic web is a web of data. The data can be stored in documents in
various ways, through for example RDFs (Resource Description Framework). RDFs
are structerd data, that can be linked to each other with information about
and relations to objects. An example of an RDF object describing two CDs can
be found in figure The aim of the semantic web is to be able to create more
advanced knowledge management systems that can, for example locate
inconsistencies in information, present the answer to a query instead of a
page where the answer can be found and answer queries where the answer is
spread over several documents.

\begin{lstlisting}[language=xml, caption=An RDF example describing two records, label=rdfexample]
<?xml version="1.0"?>
<rdf:RDF 
  xmlns:rdf="http://www.w3.org/1999/02/22-rdf-syntax-ns#"
  xmlns:cd="http://www.recshop.fake/cd#">

  <rdf:Description
  rdf:about="http://www.recshop.fake/cd/Empire Burlesque">
    <cd:artist>Bob Dylan</cd:artist>
    <cd:country>USA</cd:country>
    <cd:company>Columbia</cd:company>
    <cd:price>10.90</cd:price>
    <cd:year>1985</cd:year>
  </rdf:Description>

  <rdf:Description
  rdf:about="http://www.recshop.fake/cd/Hide your heart">
    <cd:artist>Bonnie Tyler</cd:artist>
    <cd:country>UK</cd:country>
    <cd:company>CBS Records</cd:company>
    <cd:price>9.90</cd:price>
    <cd:year>1988</cd:year>
  </rdf:Description>
</rdf:RDF>
\end{lstlisting}

\subsection{The Semantic Web - By using RDF} Resource Description Framwork
(RDF) is a general method to describe or model information that is implemented
in web resources. It is based upon the idea of making statements about
resources in the form of triples of subject-predicate-object. The subject
denotes the resource, the predicate denotes traits or aspects of the resource
and also expresses a relationship between the subject and the object. For
example: "The sky" (subject) "has the color" (predicate) "blue" (object) could
be an RDF triple.

\section{Rättsinformationsprojektet}
Lorem ipsum dolor sit amet, consectetuer adipiscing elit. Mauris
purus. 

\chapter{Method}

Lorem ipsum dolor sit amet, consectetuer adipiscing elit. Mauris
purus. 

\section{Existing system}

Lorem ipsum dolor sit amet, consectetuer adipiscing elit. Mauris
purus. 

\section{Python}

Lorem ipsum dolor sit amet, consectetuer adipiscing elit. Mauris
purus. 

\section{RDF}

Lorem ipsum dolor sit amet, consectetuer adipiscing elit. Mauris
purus. 

\part{Results}

\chapter{Analysis}

The result is a program that parse and transform downloaded laws. There’s no graphic user interface since that wasn’t the focus of the thesis. The programs main file is Controller.py and the process is started by running that as a python file in for example a terminal. 

\begin{lstlisting}[language=bash, caption=Command to run the program]
kungen@dell-desktop:~/Dropbox/Notisum/lawParse$ python Controller.py
No parameters given
Valid parameters are: DownloadAll, GenerateAll, ParseAll
kungen@dell-desktop:~/Dropbox/Notisum/lawParse$ 
\end{lstlisting}

As the above figure shows the program expects a command parameter as input. The option to choose a command when there’s only one (ParseAll) available at the moment is implemented to facilitate the implementation of other future functions as well. 

\section{RDF}

Lorem ipsum dolor sit amet, consectetuer adipiscing elit. Mauris
purus.

\section{Rättsinformationsprojektet standard}

Lorem ipsum dolor sit amet, consectetuer adipiscing elit. Mauris
purus.  

\section{Performance}

Lorem ipsum dolor sit amet, consectetuer adipiscing elit. Mauris
purus. 

\chapter{Implementation}

Lorem ipsum dolor sit amet, consectetuer adipiscing elit. Mauris
purus. 

\section{Central classes}

Lorem ipsum dolor sit amet, consectetuer adipiscing elit. Mauris
purus. 

\section{General functionality}

The help classes are used to simplify (mostly) for the Parser by defining native python types, functions with functionality adapted for this project's needs.  

\subsection{DataObjects.py}

Makes it possible to build an object model for each legal source document, by
creating simple data objects. These objects are base data types that inherit
from native python types such as unicode, list, dict. The data types have
added support for other properties that can be set when instantiated.

\subsection{Dispatcher.py}

Lorem ipsum dolor sit amet, consectetuer adipiscing elit. Mauris
purus. 

\subsection{Reference.py}

Parses plaintext and finds references to other legal source documents and
returns a list with Link-objects\footnote{A Link-object is a base data type
that inherits from 'Unicode Structure' in DataObjects.py, basically a unicode
string with a 'URI' property.}. Depending on with which properties it is
initialized, it can find different types of references (Other laws,
'Rättsfall', 'EG-lagstiftning' etc..).

\subsection{TextReader.py}

Lorem ipsum dolor sit amet, consectetuer adipiscing elit. Mauris
purus. 

\subsection{Util.py}

A few small help functions mostly related to checking and or renaming directories and files. 

\section{Unicode}

Lorem ipsum dolor sit amet, consectetuer adipiscing elit. Mauris
purus. 

\section{3rd party libraries}

Lorem ipsum dolor sit amet, consectetuer adipiscing elit. Mauris
purus. 

\section{Version control}

GitHub\footnote{GitHub is a web-based hosting service for software development
projects that use the Git revision control system. \url{http://github.com}} is
used for web hosting and revision control, the code used for this report is
open source and available to checkout from the link below. On the github page
there's also an issue tracker with the current open issues, bugs and some
enhancements.\\\\
Project: \url{https://github.com/Sup3rgnu/lawParse}\\
Checkout: \url{https://github.com/Sup3rgnu/lawParse.git}

\chapter{Conclusions and Discussions}

Lorem ipsum dolor sit amet, consectetuer adipiscing elit. Mauris
purus. 

\section{Results}

Lorem ipsum dolor sit amet, consectetuer adipiscing elit. Mauris
purus. 

\section{Future development}

Lorem ipsum dolor sit amet, consectetuer adipiscing elit. Mauris
purus. 

\subsection{RDF Database}

Lorem ipsum dolor sit amet, consectetuer adipiscing elit. Mauris
purus. 

\chapter*{Bibliography}

[1] Ted Talk video about Linked Data with Tim Berners-Lee \url{http://www.ted.com/talks/tim_berners_lee_on_the_next_web.html}


\appendix
\addappheadtotoc
\chapter{RDF}\label{appA}

\begin{figure}[ht]
\begin{center}
And here is a figure
\caption{\small{Several statements describing the same resource.}}\label{RDF_4}
\end{center}
\end{figure}

that we refer to here: \ref{RDF_4}
\end{document}
