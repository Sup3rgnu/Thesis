\begin{foreignabstract}{swedish}
Detta projekt har utförts som ett examensarbete på skolan för datavetenskap och kommunikation på Kungliga Tekniska Högskolan i Stockholm (KTH) på uppdrag av Notisum AB. Projektets syfte är att undersöka huruvida Notisums nuvarande process för att formatera nerladdade juridiska dokument kan förbättras genom en modernare process. En del av processen är att märka upp datat i dokumenten enligt Rättsinformationsprojektets föreslagna standard. Detta ska bidra till en mer semantisk webb där all data är meningsfull för både människor och datorer. Resource Description Framework (RDF), ett ramverk som används för att beskriva eller modellera information från webbkällor har använts för att ta fram en prototyp för att formattera dokument (lagar) från Svensk Författnigsamling i språket Python. I arbetet ingick även forskning kring fenomenet “Den Semantiska Webben”, vad den innbär och hur man kan gå till väga för att uppfylla dess krav. Slutligen föreslås rekommendationer för hur Notisum kan gå vidare med användandet av den  framtagna metoden för att märka upp lagar på ett meningsfullt sätt och även vissa förbättringar som skulle kunna genomföras. 
\end{foreignabstract}
