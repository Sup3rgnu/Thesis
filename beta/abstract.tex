\begin{abstract}
This project has been carried out as a thesis project at the School of Computing at the Royal Institute of Technology in Stockholm (KTH) on behalf of Notisum AB. The project aims to investigate whether Notisum’s current process to parse and format downloaded legal documents can be improved by a more modern process. Part of the process is to mark up the data in the legal documents according to Rättsinformationsprojektets proposed standard. This will contribute to a more semantic web where all data is meaningful to both humans and computers. The Resource Description Framework (RDF), is a framework that is used to describe or model information from web sources, have been used to produce a prototype to parse and format documents (mainly laws) from Swedish Constitution (SFS) in the programming language Python. This work also included research into the phenomenon of "The Semantic Web", what it prejudge and how to proceed in order to meet its requirements. Finally, it proposes recommendations on how Notisum can proceed with the use of the developed method for parsing legislations in a meaningful way and also some improvements that could be implemented.
\end{abstract}